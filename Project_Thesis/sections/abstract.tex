The IllustrisTNG project is an ongoing series of state-of-the-art cosmological magneto-hydrodynamical galaxy formation simulations. It is one of the most complete simulations in the field of astrophysics, and allows for exciting new possibilities in the study of galaxy formation and evolution. In this project, the publicly available group catalog data output from the TNG-100 simulation for $z=0$ is investigated to check its efficiency in modelling known galaxy properties. The properties and scaling relations that are studied are the stellar to halo mass relation, the Tully-Fisher relation, the projections of the Fundamental Plane, including the Faber-Jackson relation, and the galaxy color bimodality. Galaxies are split into early and late type galaxies which are analyzed separately when relevant. The oservational data which TNG is compared against is mostly taken from the SAMI Galaxy survey. Only using the group catalogs was a limiting factor in fairly comparing properties. Still, with some deviations, the TNG data was found to generally reproduce the relations and properties found in the observational data.