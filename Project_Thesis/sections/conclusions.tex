
A study of the data output from the IllustrisTNG simulation TNG-100 for $z=0$ was carried out to see how well it reproduces known galaxy properties. The property relations studied were the stellar to halo mass relation, the Tully-Fisher relation, the Faber-Jackson relation and the Fundamental Plane as well as the galaxy color bimodality. Only galaxies with stellar mass greater than $10^9 M_{\odot}$ were used in the study, to ensure sufficiently resolved structure in the inner parts of the galaxies. All the property values were taken from the publicly available data catalogs. For comparison purposes the observational data from the SAMI galaxy survey were chosen, which also are publicly available as data catalogs.  

For the stellar to halo mass relation, TNG produces galaxies with too high stellar mass at both low and high halo masses compared with results from abundance matching. The general trend and characteristic mass is however similar. 

The Tully-Fisher relation for TNG is a clear power law with small scatter, the same as found empirically. It has however a steeper slope than that found for SAMI. There are still many opportunities to improve the comparison, and further study is required to investigate this. 

Looking at early type galaxies, the Faber-Jackson relation for TNG has a similar slope to that of SAMI, but there is a systematic shift towards lower velocity dispersions for the simulation compared to the observed data. This is again something which could be further studied in future work, specifically by calculating the velocity dispersions for TNG in the inner parts of the galaxies and only using stellar particles. When looking at the other projections of the Fundamental Plane, the effective radius for TNG is lower than that for SAMI. Here the different definitions of effective radii plays a significant role, and in future work it would be beneficial to try to use the most similar definitions.

Finally, the color bimodality shows a clear separation into a red and blue group in the g-i color-mass diagram. It was found that TNG galaxies were on average a little bluer than SAMI galaxies. However, here the much smaller data sample in SAMI becomes apparent. There is also the differing methods of assigning the galaxy morphology types to take into account.

In general, on most of the properties studied here, TNG results show good agreement with observations, but the limitations of only working with the premade group catalog is apparent. In future works, I would like to study entire snapshots instead. This is a much more time consuming approach, but the benefits are a greater control over exactly how properties are calculated, and the possibility of looking at properties which are not covered in the group catalog. Also, SAMI data release 3 is in the process of being made public right now, and getting a larger observational sample for comparison would be great.

In conclusion, IllustrisTNG is a powerful tool for studying galaxy formation and evolution. Cosmological hydrodynamical simulations are getting more and more powerful as computer power increases and we get a better understanding of the physics that governs the evolution of our Universe. There is of course always room for improvements, but this is what makes the field of astrophysics such an exciting field of study.