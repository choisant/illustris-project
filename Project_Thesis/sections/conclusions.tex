
A study of the data output from the IllustrisTNG simulation TNG-100 for redshift 0 was carried out to see how well it reproduces known galaxy properties. The properties studied were the stellar to halo mass relation, the Tully-Fisher relation, the Faber-Jackson relation and fundamental plane as well as the galaxy color bimodality. Only galaxies with stellar mass greater than $10^9 M_{\odot}$ were used in the study, to ensure sufficient resolved structure in the inner parts of the galaxies. All the property values were taken from the publicly available data catalogs. For comparison purposes the observational data from the SAMI galaxy survey were chosen, which also are publicly available as data catalogs.  

For the stellar to halo mass relation, TNG produces galaxies with too high stellar mass at both low and high halo masses compared with results from abundance matching. The general trend and characteristic mass is however similar. 

The Tully-Fisher relation for TNG has a steeper slope than that found for SAMI. There is however many deficits in the comparison, and further study is required to investigate this. 

Looking at early type galaxies, the Faber-Jackson relation for TNG has a similar slope to that of SAMI, but there is a systematic shift towards lower velocity dispersions for the simulation compared to the observed data. This is again something which could be further studied in future work, specifically by calculating the velocity dispersions for TNG only in the inner parts of the galaxies and only using stellar particles. When looking at the other projections of the fundamental plane, the effective radius for TNG is lower than that for SAMI. Here the different definitions of effective radii plays a significant role, and in future work it would be beneficial to try to use the most similar definitions.

The color bimodality of galaxies was also investigated, and it was found that TNG galaxies were on average bluer than SAMI galaxies. However, here the much smaller data sample in SAMI becomes apparent. Again there is also the differing methods of assigning the galaxy morphology types to take into account.

////

In conclusion, IllustrisTNG is a powerful tool for studying galaxy formation and evolution. On most of the properties studied here, it shows excellent agreement to observations. 
In future works, I would like to study entire snapshots instead of the premade data catalogues. In particular, it would be interesting to look at the velocity profiles of galaxies in more detail. Also, SAMI data release 3 is being released right now. Getting a larger observational sample for comparison would be great.

Cosmological hydrodynamical simulations are getting more and more powerfull as computer power increases and we get a better understanding of the physics that govers the evolution of our universe. There is of course always room for improvements, and so the field of astrophysics will just keep evolving as we learn and grow.