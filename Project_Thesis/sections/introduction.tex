
\noindent
\subsection{Motivation}
The field of astrophysics is a relatively young field of study compared to most other disciplines of science, but in many ways it is also the most fundamental. From the tiniest quantum fluctuations at the beginning of time, to galaxy clusters, astrophysicists have to cover a range of magnitudes from the smallest particles discovered to the largest structures we know about. 

In this project, galaxies are the focus of study. Theories for how galaxies formed and evolved since the Big Bang have been proposed since they were first discovered, and as new data and new physics emerge, new theories take over for old ones. The model that has been established as the one best able to explain our observations of the Universe is the Lambda cold dark matter ($\Lambda$CDM) model \parencite{white}. In this model, the energy in the universe is made up of about 75 percent dark energy (the so-called vacuum energy that is pushing the expansion of the Universe), 21 percent dark matter and about 4 percent baryonic matter. 

There are many theories for what dark matter is \parencite{Boveia2018}, but what we do know is that cosmological models require the presence of dark matter to reproduce the structures seen today. Dark matter does not interact with baryonic matter in any noticeable way except through gravity. In the $\Lambda$CDM model of our Universe, galaxies are located in the center of dark matter halos (or just halos), which extend much further than the actual visible galaxy. Many of the properties of the galaxies are linked to its host halo. These, along with several other galaxy properties, are the main focus of this project report.

Hydrodynamical cosmological simulations have been around since the 1980s, starting as N-body simulations of only dark matter with a set of initial conditions \parencite{Frenk1983}. As computers became more powerful, and physicists learned more about the complicated physics of galaxies, the simulations started to incorporate stars, gas and other baryonic components. The number of particles that can be resolved within a given space has increased tremendously, and newer projects such as the Illustris and EAGLE simulations have pushed the boundaries of modern astrophysics. IllustrisTNG is the new and improved version of the Illustris simulation, with the first result-papers being published in 2017, and more data being produced still. It increases the resolution, size and amounts of physics included, to produce the largest, most detailed simulated universe to this date. 

In this report, the data from the IllustrisTNG simulations will be compared against observational data, to determine whether it manages to reproduce known galaxy properties. The data used is from the readily available data catalogues, and so this will also be a good way to check the usefulness of this resource which can easily be downloaded and studied by people without much programming experience.


\subsection{The structure of this report}
In this report ....
The theory section explains the physics of the main galaxy property relations that are covered in this report. It also hopefully serves as a sort of glossary and explanation for many of the (sometimes confusing) astrophysical terms. In the method part of the report, it is explained how the simulation and observation data is filtered and edited to be comparable. The results and discussion section covers the comparison of the data, while the conclusions section sums up what was learned from the project and looks to the future for what should be studied next.