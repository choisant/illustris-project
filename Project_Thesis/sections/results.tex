\subsection{FJ relation and the FP}

\begin{figure}
    \centering
    \includegraphics[width=\textwidth]{images/faber_jackson.png}
    \caption{Early type galaxies for both TNG and SAMI.}
    \label{FJ_results}
\end{figure}

The velocity dispersion as function of stellar mass can be seen in Figure \ref{FJ_results}. The trend for the TNG-100 data is a clear power law as expected from the FJ relation. Compared to the observational data, the simulation data shows lower $\sigma$ values, by about 0.1-0.2 dex. This could be explained by the fact that the $\sigma$ in the TNG galaxies is averaged across all particles, across the whole size of the subhalo. In general, gas has a lower $\sigma$ than stars and dark matter, so this could push the total $\sigma$ down. However, in early-type galaxies there is little gas so the impact would be expected to be small. The fact that $\sigma$ is found by averaging across the entire subhalo would include particles further out than for the SAMI data in which the velocity dispersion is averaged inside the effective radius ($\sigma_{e}$) is used. Other studies have also found that simulations tend to get lower values for $\sigma$ \parencite{Sande2018}, so this might also just be a limitation of the simulations.