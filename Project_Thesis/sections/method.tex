\subsection{IllustrisTNG}
%Add footnote with IllustrisTNG webpage
IllustrisTNG \footnote{https://www.tng-project.org/} is the follow-up project after the success of the Illustris simulations. It is a huge project, built upon a magneto-hydrodynamical cosmological simulation code with added physical processes on a subgrid level \parencite{Weinberger2016}. Adding physical processes like gas radiation, star formation, stellar feedback through supernova explosions, supermassive blackhole accretion and magnetic fields are essential to model galaxy formation and evolution, and allows a much better comparison to reality. The data output from the simulations is extensive, and are not meant to be analysed all in one go, but rather through a series of analyses, each targeting a specific scientific question. 


\subsubsection{The simulations}
The IllustrisTNG project includes 18 different simulations with varying resolutions, spatial size and included physics. There are three main simulations that differ in volume and resolution, and the details of these are summed up in Table \ref{TNG}. Each of the main simulations have been run at three different resolution levels, which makes it possible to study how changing only the resolution in a given simulation affects the outcome. TNG100 has a physical box volume of $110.7^3 \, $Mpc, and a baryonic particle resolution of $1.4 \times 10^6 M_{\odot}$, while the TNG300 simulation has a volume of $302.6^3 \, $Mpc and a baryonic particle resolution of $1.1 \times 10^7 M_{\odot}$. The TNG50 data is actually not yet available, but it is expected soon, and provides a much higher resolution in a smaller box size. In this project, a large statistical sample of galaxies was needed, as well as detailed structure of the inner part of the galaxies to calculate the different properties, so the TNG100 simulation was the ideal middle ground with respect to size and resolution. The TNG100-1 simulation data has been used throughout the project, which is the highest available resolution for TNG100.

\begin{table}
\begin{center}
\begin{tabular}{ l| c c c c c } 
 \hline
 \hline
   &  Volume [$Mpc^3$] & $N_{DM}$ & $m_{DM}$ [$M_{\odot}$] & $m_{baryon}$ [$M_{\odot}$] \\
  \hline
 TNG50 & $51.7^3$ & $2163^3$ & $4.5 \times 10^5 $ & $8.5 \times 10^4 $ \\ 
 TNG100 & $110.7^3$ & $1820^3$ & $7.5 \times 10^6 $ & $1.4 \times 10^6 $  \\ 
 TNG300 & $302.6^3$ & $2500^3$ & $5.9 \times 10^7 $ & $1.1 \times 10^7 $  \\ 
 \hline 
 \end{tabular}
\end{center}
\caption{The simulation details for the three main TNG simulations. $N_{DM}$ is the amount of dark matter particles. $m_{DM}$ and $m_{baryon}$ is the mass of the dark matter and baryonic particles, respectively.}
 \label{TNG}
\end{table}

\subsubsection{Data cataloges}
All the Illustris-TNG data is publically available online at the TNG webpage. The data products that are available for each simulation are snapshots, group catalogs and merger trees as well as some supplementary data sets. There are 100 snapshots for each run, which are taken at specific redshifts. They include all the particles in the whole volume of the simulation, with 20 of them including all the particle fields for each particle as well. There are five different particle types, and each particle has its properties stored as particle fields. These fields include information like position, kinematic data and atomic/chemical composition. 


The group catalogs provide a convenient way to quickly access already calculated properties of the different halos and subhalos instead of dealing with at all the particles in a snapshot. This saves a lot of time and effort, but gives the user less control over what can be analysed. In future work, it might be interesting to do the calculations directly from the snapshots myself. There is one group catalog for each snapshot, and this includes two types of objects, Friends-of-Friends (FoF) and Subfind. The FoF catalog contains all the DM halos, and the Subfind catalog contains all the subhalos for each halo. Each subhalo has a parent halo, and the largest subhalo in each halo is the central subhalo. The merger trees data products contain the merger history of each subhalo.

This project makes use of the group catalogs for the $z = 0$ snapshot in TNG100-1, as we want to compare the output data to the present universe.

\subsubsection{Cuts made in the data} %badtitle

The TNG documentation recommends filtering out all subhalos that are flagged with the $SubhaloFlag$ field, and so these were cut from the data. These are most probably subhalos of non-cosmological origin, and so should not be considered real galaxies.

For most of the relations covered in this project, it is desirable to only use the central galaxies in each halo. This is because satellite galaxies are more affected by their environment, which in turn affects the kinematic and structural properties of the galaxy. This will naturally lead to a scatter in the galaxy scaling relations that are being studied, which central galaxies will not display. The FoF catalog contains the index for the largest subhalo in each halo, so combining this information with the Subfind catalog allows one to create a subset of the data that contains only the central galaxies.

Only galaxies with stellar mass greater than $10^9 M_{\odot}$ were included. This is because smaller galaxies will have fewer stellar particles, and thus their structure is not necessarily reliably resolved.

\begin{equation}
    M_{gas}/M  = f
\end{equation}

For $f > 0.1$, the galaxy is classified as late type, while for $f< 0.1$, the galaxy is classified as early type.

\subsection{Observational data}
Blabla comparing important integrity of dfasdf
It is desirable to use the same observational data when comparing different scaling relations, however it has not been possible to do that. This is because we are analysing such different problems as stellar-to-halo mass and SMBH relations, which require very different kind of measurements. A compromise has been to use one main survey for the kinematic scaling relations. //fill in when you know more about this //

\subsubsection{SAMI Galaxy Survey}
Sydney – Australian Astronomical Observatory Multi-Object Integral Field Spectrograph (SAMI) is mounted on the Anglo-Saxan telescope in Australia. The SAMI Galaxy Survey \footnote{https://sami-survey.org/} is a spectroscopic survey of a large sample of galaxies in the nearby universe ($z < 0.113$). The survey was started in 2013, and ended in 2018. There have been two major data releases, with the newest being Data Release Two (DR2) \parencite{Scott2018}. DR2 includes data for 1559 galaxies, which are about 50 \% of the full galaxy survey. The data products available are IFS data cubes and 2D maps, as well as catalogue data. Analysing data cubes and 2D maps falls outside the scope of this product, so catalogue data is used where possible. Some data is not available in the catalogues, but the direct results from research using the SAMI data has then been used.

%%morphology here? or in properties?
\subsubsection{Other data sets}
SHMR
SMBH

\subsection{Calculating properties}


\subsubsection{Separating out early and late type galaxies}
As several of the relations studied in this project relate to the morphological type of the galaxies, it is interesting to filter out early and late type galaxies to study separately. This can be done in different ways, and in many studies of TNG several criteria for classification have been chosen. In this case, the fraction of gas inside the effective radius of each galaxy has been chosen as the single criteria for classification. Including a criteria for star formation rate did not significantly change the outcome, so it was determined to keep the selection process simple.

In the SAMI DR2, the galaxy morphology is determined visually. They are classified into four different categories: ellipticals, S0, Sa/Sb and Sc/Sd/irregulars. See Figure \ref{hubble} in section 2.2 for a visualisation of the different galaxy classifications.

\subsubsection{Circular velocities}
To compare the simulation data with observational data for rotational velocities, calculated circular velocities are used. The Subhalo field $SubhaloVMax$ gives the maximum value for the spherically averaged rotation curve. A the rotational curves are nearly flat for large enough radii, it is not very important at which radius the observational rotational velocity is chosen. For the SAMI-data velocity curves were only available as 2D maps and not catalog data. So the best fit from \parencite{Bloom2017}, which already did the analysis of the 2D velocity maps, has been used. They used the rotational velocity at $2.2\, R_e$, which should lay well into the flat regime of the velocity curve, and coincide well with the maximum velocity.

%maximum velocity,velocity curves...2.2R_e
\subsubsection{Effective radius}
In observational data, galaxy sizes are always projected sizes, as they are derived from 2D pictures. A common measure of the size of a galaxy is the effective radius, which is the radius within which is contained half the light of the galaxy. This quantity depends then on the analysis and quality of the 2D profiles, and may not be able to include all the light in a galaxy in the way that we can ensure that for TNG. It also depends on which band the measurements are made in, as different bands will capture different parts of the galaxy.

For TNG data, the $SubhaloHalfmassRadStellar$ field has been used. The half-mass radius is the radius within which half the stellar mass is found. This is not a projected quantity, so it can be considered the 3D half-mass radius. This value is in generally higher for a given mass up to $M_{*} < 10^10.5$ than the 2D projected haøf-light radius, as seen in \parencite{Genel2017}.

The SAMI catalog data takes the values for the effective radius from the GAMA Sérsic catalogue \parencite{Kelvin2012}. The half-light radius is defined as the semi-major axis effective radius. To convert to circular radius, the definition of ellipticity $\epsilon$ is used:

\begin{equation}
   r_{e, circ} = r_{e,sm}\sqrt{(1-\epsilon)}
\end{equation}


%half mass 3D radius vs 2D projected half light radius: Lovell2019/Genel2018
%also circular radii

\subsubsection{Dynamical mass??}